% Template for Cogsci submission with R Markdown

% Stuff changed from original Markdown PLOS Template
\documentclass[10pt, letterpaper]{article}

\usepackage{cogsci}
\usepackage{pslatex}
\usepackage{float}
\usepackage{caption}

% amsmath package, useful for mathematical formulas
\usepackage{amsmath}

% amssymb package, useful for mathematical symbols
\usepackage{amssymb}

% hyperref package, useful for hyperlinks
\usepackage{hyperref}

% graphicx package, useful for including eps and pdf graphics
% include graphics with the command \includegraphics
\usepackage{graphicx}

% Sweave(-like)
\usepackage{fancyvrb}
\DefineVerbatimEnvironment{Sinput}{Verbatim}{fontshape=sl}
\DefineVerbatimEnvironment{Soutput}{Verbatim}{}
\DefineVerbatimEnvironment{Scode}{Verbatim}{fontshape=sl}
\newenvironment{Schunk}{}{}
\DefineVerbatimEnvironment{Code}{Verbatim}{}
\DefineVerbatimEnvironment{CodeInput}{Verbatim}{fontshape=sl}
\DefineVerbatimEnvironment{CodeOutput}{Verbatim}{}
\newenvironment{CodeChunk}{}{}

% cite package, to clean up citations in the main text. Do not remove.
\usepackage{apacite}

% KM added 1/4/18 to allow control of blind submission


\usepackage{color}

% Use doublespacing - comment out for single spacing
%\usepackage{setspace}
%\doublespacing


% % Text layout
% \topmargin 0.0cm
% \oddsidemargin 0.5cm
% \evensidemargin 0.5cm
% \textwidth 16cm
% \textheight 21cm

\title{Measuring a Basic Developmental Vocabulary Across Many Languages}


\author{{\large \bf George Kachergis (kachergis@stanford.edu)} \\ Department of Psychology, 450 Jane Stanford Way \\ Stanford, CA 94305 USA \AND {\large \bf Michael C. Frank (mcfrank@stanford.edu)} \\ Department of Psychology, 450 Jane Stanford Way \\ Stanford, CA 94305 USA}

\newlength{\cslhangindent}
\setlength{\cslhangindent}{1.5em}
\newenvironment{CSLReferences}%
  {}%
  {\par}

\begin{document}

\maketitle

\begin{abstract}
Early language skill is predictive of later life outcomes, and is thus
of great interest to developmental psychologists and clinicians. The
Communicative Development Inventories (CDIs), including a
parent-reported inventory of early-learned vocabulary items, has proven
to be a valid and reliable instrument for measuring children's early
language skill. CDIs have been adapted to dozens of languages, and
cross-linguistic comparisons thus far show both consistency and
variability in language acquisition trajectories. Here, we use
item-response theory models to examine the psychometric properties of
translation-equivalent concepts that have been included on CDIs in
several languages, with the goal of identifying a short list of concepts
that are of approximately equal difficulty across the majority of the
languages. Using a list of XX items, we test how well this

\textbf{Keywords:}
early language learning; CDI; psychometrics; cross-linguistic
comparison; Swadesh vocabulary;
\end{abstract}

\hypertarget{introduction}{%
\section{Introduction}\label{introduction}}

Children's early language skill is predictive of educational outcomes,
other developmental milestones, etc.

Measuring children's early language is important for parents, clinicians
and researchers. The MacArthur-Bates Communicative Development
Inventories {[}CDIs; L. Fenson et al. (2007){]} are a set of parent
report forms that offer a holistic assessment of children's productive
and receptive language skills. CDIs are low-cost to administer and
produce reliable and valid estimates of early vocabulary and other
aspects of early language (Larry Fenson et al., 1994). The CDIs offer
more comprehensive data than a short interaction in the lab with a child
(L. Fenson et al., 2000), because they ask parents to report on
vocabulary comprehension as well as production, and other milestones,
such as communicative gesture use and use of word combinations.
Vocabulary size is assessed via a checklist format, which allows
caregivers to quickly scan and recognize words their child produces or
understands, rather than relying on recall alone. Because of these
properties, CDI forms have been adapted to dozens of languages. Data
from CDIs are archived in a central repository {[}Wordbank; Frank,
Braginsky, Yurovsky, \& Marchman (2017){]}, and insights from these data
have been used to inform theories of early language learning (Frank,
Braginsky, Yurovsky, \& Marchman, 2021).

Although CDIs measure a variety of other constructs related to early
language, our focus here is on vocabulary assessment. Across languages,
measures of vocabulary on the CDIs are very tightly correlated with
other aspects of early language like gesture and grammatical competence
(Bates et al., 1994; Frank et al., 2021). From the perspective of the
CDI, it is justifiable to say that the language system is ``tightly
woven'' (Frank et al., 2021), meaning that precise measures of early
vocabulary provide a good proxy measurement of the language system as a
whole.

In American English and Mexican Spanish (two of the original languages
in which the CDIs were developed), there are two long-form CDI
instruments that focus on different ages. The 396-item vocabulary
checklist on the CDI: Words \& Gestures (CDI:WG) was designed for
children ages 8--16 months and measures both comprehension and
production. The 680-item vocabulary checklist on the CDI: Words \&
Sentences (CDI:WS) targets children ages 16--30 months, and includes
nearly all of the items from the CDI:WG form, but only measures
production. All CDI forms include words from a range of semantic and
syntactic categories. For example, the CDI:WS form is comprised of 22
semantic categories representing common early-learned words, including
nouns (subdivided into e.g., Body Parts, Toys, and Clothing), action
words (e.g.~verbs), descriptive words (e.g.~adjectives), and
closed-class words such as pronouns.

Although the CDIs have many advantages, one clear drawback is the long
length of the vocabulary checklists, which make them time-consuming for
parents to complete. The length of the CDIs also makes it difficult to
include them in studies requiring a battery of tasks, as is often the
case in both clinical and research settings. Due to these challenges,
there have been a variety of efforts to create shortened versions of the
CDI. The 100-item short-form CDIs (L. Fenson et al., 2000, 2007) are
derived from the 680-item CDI:WS form, with items selected based on
difficulty and item-to-full score correlations, while also attempting to
represent the diversity of semantic and linguistic categories. While the
scores of the short-form CDI:WS are highly correlated with scores on the
full CDI:WS, there is evidence for a ceiling effect for children older
than 27 or 28 months of age.

The basis of CAT is item-response theory modeling {[}IRT; Embretson \&
Reise (2013){]}, a technique for the analysis of test data that allows
the inference of both the abilities of test takers and the difficulty
(and other information) of individual test questions along shared and
standardized dimensions. CAT models use this item information --
typically extracted from a larger dataset collected via standard testing
methods -- to select questions of the appropriate difficulty for a
particular test-taker. An individual CAT includes a number of
components, including the bank of possible items and their difficulties,
as well as an algorithm that uses the responses received thus far to
choose the next item to give to a test taker, and a rule for when to
stop (e.g., after a fixed number of items or after a desired precision
has been reached).

Previous work has applied CAT and related techniques to CDI forms,
leveraging the availability of large datasets from previous CDI studies
where parents filled out the full forms (Chai, Lo, \& Mayor, 2020;
Makransky, Dale, Havmose, \& Bleses, 2016; Mayor \& Mani, 2019). For
example, Makransky et al. (2016) used IRT models fit to normative data
from the CDI:WS to develop CAT versions (L. Fenson et al., 2007). They
conducted a simulation study comparing full scores on the CDI to
different fixed-length CAT versions (with 5--400 items). Scores from a
CAT of only 50 items had a correlation of \(r=0.95\) with the full CDI.
However, for the youngest age group (16--18 month-olds), the correlation
with the full CDI was somewhat lower (\(r=0.87\)).

Our goal in the current work is to\ldots{} In particular, our
contributions are:

We will first introduce the candidate Item Response Theory (IRT) models,
fit them to the datasets, and assess which model is appropriate as the
basis for the adaptive CDI. Next, we will conduct various CAT
simulations to assess the effects of different design choices on test
performance, and choose a set of preferred CAT parameters. Finally, we
report results from a validation study. We end by discussing the
strengths and weaknesses of our approach.

\hypertarget{methods}{%
\section{Methods}\label{methods}}

\hypertarget{item-response-theory-models}{%
\subsection{Item Response Theory
Models}\label{item-response-theory-models}}

A variety of IRT models targeting different types of testing scenarios
have been proposed (see Baker, 2001 for an overview), but for the
dichotomous responses that parents make for each item (word) regarding
whether their child can produce that word, we will use the popular
2-parameter logistic (2PL) model that we have previously found is best
justified for CDI data (see \textbf{Kachergis2022?}).

The 2PL model jointly estimates for each child \(j\) a latent ability
\(\theta_j\) (here, language skill), and for each item \(i\) two
parameters: the item's difficulty \(b_i\) and discrimination \(a_i\),
described below. In the 2PL model, the probability of child \(j\)
producing a given item \(i\) is

\[P_{i}(x_i = 1 | b_{i},a_{i},\theta_j ) = \frac{1}{1 + e^{-D a_{i}(\theta_j - b_i )}}\]

Thus, children with high latent ability (\(\theta\)) will be more likely
to produce any given item than children with lower latent ability, and
more difficult items will be produced by fewer children (at any given
\(\theta\)) than easier items. The discrimination (\(a_i\)) modifies the
slope of the logistic (in the classic Rasch or 1PL model, the slope is
always 1):

If the authors' names are included in the sentence, place only the year
in parentheses, as in (\textbf{NewellSimon1972a?}), but otherwise place
the entire reference in parentheses with the authors and year separated
by a comma (\textbf{NewellSimon1972a?}). List multiple references
alphabetically and separate them by semicolons
(\textbf{ChalnickBillman1988a?}; \textbf{NewellSimon1972a?}).

\hypertarget{datasets}{%
\subsection{Datasets}\label{datasets}}

\begin{table}[H]
\centering
\begin{tabular}{lrrrr}
  \hline
Language & WS items & WS N & WG items & WG N \\ 
  \hline
Norwegian & 731 & 9304 & 395 & 2926 \\ 
  English (American) & 680 & 8645 & 396 & 4130 \\ 
  Danish & 725 & 3714 & 410 & 2398 \\ 
  Portuguese (European) & 639 & 3012 & 293 & 1314 \\ 
  Turkish & 711 & 2422 & 418 & 1115 \\ 
  Mandarin (Taiwanese) & 696 & 1897 & 354 & 757 \\ 
  Spanish (Mexican) & 680 & 1853 & 428 & 833 \\ 
  English (Australian) & 558 & 1520 &  &  \\ 
  Korean & 641 & 1376 & 284 & 618 \\ 
  Cantonese & 804 & 1295 &  &  \\ 
  German & 588 & 1181 &  &  \\ 
  Slovak & 609 & 1066 & 307 & 657 \\ 
  Mandarin (Beijing) & 799 & 1056 & 227 & 230 \\ 
  Russian & 728 & 1037 & 424 & 768 \\ 
  French (Quebecois) & 664 & 929 & 366 & 598 \\ 
  Swedish & 710 & 900 & 341 & 464 \\ 
  Spanish (Argentinian) & 699 & 784 &  &  \\ 
  Italian & 670 & 752 & 408 & 648 \\ 
  French (French) & 690 & 665 & 212 & 222 \\ 
  Spanish (European) & 588 & 593 & 277 & 412 \\ 
  Hebrew & 605 & 518 & 439 & 62 \\ 
  Latvian & 723 & 500 & 402 & 183 \\ 
  Czech & 553 & 493 &  &  \\ 
  Croatian & 717 & 377 & 380 & 250 \\ 
  Hungarian & 802 & 363 &  &  \\ 
  Dutch & 704 & 303 & 160 & 317 \\ 
  Greek (Cypriot) & 815 & 176 &  &  \\ 
  Spanish (Peruvian) & 600 & 105 & 112 & 87 \\ 
  Kigiriama & 696 & 100 & 260 & 132 \\ 
  English (Irish) & 660 & 99 &  &  \\ 
  Irish & 691 & 99 &  &  \\ 
  Kiswahili & 705 & 90 & 216 & 51 \\ 
  Finnish & 581 & 70 &  &  \\ 
  Persian & 558 & 50 & 367 & 115 \\ 
   \hline
\end{tabular}
\caption{Number of CDI:WG and CDI:WS items and subjects (N) per language.} 
\end{table}

We report analyses for twenty-four languages from Wordbank (Frank et
al., 2017), comprising production data from CDI:WG vocabulary
checklists. The languages are: British Sign Language, Croatian, Danish,
Dutch, English (U.S.), French (France), French (Quebec), Hebrew,
Italian, Kigiriama, Kiswahili, Korean, Latvian, Mandarin (Taiwan),
Norwegian, Persian, Portuguese (Portugal), Russian, Slovak, Spanish
(Chile), Spanish (Mexico), Spanish (Spain), Swedish, and Turkish.

\hypertarget{participants}{%
\subsubsection{Participants}\label{participants}}

The CDI:WG production dataset consists of the combined Wordbank
production data for 21055 children aged 12-18 months on 9805 items
across 28 forms. The CDI:WS production dataset consists of the combined
Wordbank production data for 47344 children aged 16-30 months on 23020
items across 34 forms. Figure 1C and 1D shows children's production
scores vs.~age for the CDI:WG and CDI:WS datasets respectively. Note
that the socioeconomic distributions of these datasets are not matched.
(See Frank et al. (2021) for a discussion of possible effects of
socioeconomic status on vocabulary development.)

\hypertarget{instruments}{%
\subsubsection{Instruments}\label{instruments}}

The 680-item vocabulary checklist of the English CDI:WS form is
organized into 22 semantic categories (e.g., furniture, games and
routines, people). The Spanish CDI:WS vocabulary checklist consists of
680 words, organized into 23 semantic categories. The English CDI:WG
vocabulary checklist is comprised of XXX of the easier vocabulary items
from the English CDI:WS, and the Spanish CDI:WG is a subset of XXX of
the items from the Spanish CDI:WS.

When a CDI:WG form was administered, caregivers were asked to indicate
for each vocabulary item whether their child 1) understands that word
(``comprehends'') or 2) both understands and says (``produces'') that
word. Leaving the item blank indicates that the child neither
comprehends nor produces that word. When a CDI:WS forms was
administered, caregivers were asked to indicate for each vocabulary item
on the instrument whether or not their child can recognizably produce
(say) the given word.

The current study solely investigates production, thus ``produces''
responses were coded as 1 and all other responses were coded as 0. Our
datasets consist of a dichotomous-valued response matrix for each
language, of size \(N\) subjects \(\times\) \(W\) words.

\hypertarget{results}{%
\section{Results}\label{results}}

All models, simulations, and other materials are available on
OSF\footnote{OSF repository: \url{https://osf.io/XXX/}.}.

\begin{CodeChunk}
\begin{figure}[H]

{\centering \includegraphics{figs/image-1} 

}

\caption[One column image]{One column image.}\label{fig:image}
\end{figure}
\end{CodeChunk}

\begin{CodeChunk}
\begin{figure}[H]

{\centering \includegraphics{figs/plot-1} 

}

\caption[R plot]{R plot}\label{fig:plot}
\end{figure}
\end{CodeChunk}

\hypertarget{discussion}{%
\section{Discussion}\label{discussion}}

\hypertarget{acknowledgements}{%
\section{Acknowledgements}\label{acknowledgements}}

We would like to thank all of the contributors to Wordbank, from the
researchers who created and adapted the CDIs to those who collected the
data (as well as the participants), to those who have created and
maintained Wordbank over the years.

\hypertarget{references}{%
\section{References}\label{references}}

\setlength{\parindent}{-0.1in} 
\setlength{\leftskip}{0.125in}

\noindent

\hypertarget{refs}{}
\begin{CSLReferences}{1}{0}
\leavevmode\vadjust pre{\hypertarget{ref-Baker2001}{}}%
Baker, F. B. (2001). \emph{The basics of item response theory}. ERIC.

\leavevmode\vadjust pre{\hypertarget{ref-bates1994}{}}%
Bates, E., Marchman, V., Thal, D., Fenson, L., Dale, P., Reznick, J. S.,
\ldots{} Hartung, J. (1994). Developmental and stylistic variation in
the composition of early vocabulary. \emph{Journal of Child Language},
\emph{21}(1), 85--123.

\leavevmode\vadjust pre{\hypertarget{ref-chai2020}{}}%
Chai, J. H., Lo, C. H., \& Mayor, J. (2020). A {B}ayesian-inspired item
response theory-based framework to produce very short versions of
{M}ac{A}rthur-{B}ates {C}ommunicative {D}evelopment {I}nventories.
\emph{Journal of Speech, Language, and Hearing Research}, \emph{63}(10),
3488--3500.

\leavevmode\vadjust pre{\hypertarget{ref-embretson2013}{}}%
Embretson, S. E., \& Reise, S. P. (2013). \emph{Item response theory}.
Psychology Press.

\leavevmode\vadjust pre{\hypertarget{ref-fenson1994}{}}%
Fenson, Larry, Dale, P. S., Reznick, J. S., Bates, E., Thal, D. J.,
Pethick, S. J., \ldots{} Stiles, J. (1994). Variability in early
communicative development. \emph{Monographs of the Society for Research
in Child Development}, i--185.

\leavevmode\vadjust pre{\hypertarget{ref-Fenson2007}{}}%
Fenson, L., Marchman, V. A., Thal, D. J., Dale, P. S., Reznick, J. S.,
\& Bates, E. (2007). \emph{{M}ac{A}rthur-{B}ates {C}ommunicative
{D}evelopment {I}nventories: User's guide and technical manual (2nd
ed.)}. Baltimore, MD: Brookes.

\leavevmode\vadjust pre{\hypertarget{ref-Fenson2000}{}}%
Fenson, L., Pethick, S., Renda, C., Cox, J. L., Dale, P. S., \& Reznick,
J. S. (2000). Short-form versions of the {M}ac{A}rthur {C}ommunicative
{D}evelopment {I}nventories, \emph{21}, 95--116.

\leavevmode\vadjust pre{\hypertarget{ref-frank2017}{}}%
Frank, M. C., Braginsky, M., Yurovsky, D., \& Marchman, V. A. (2017).
Wordbank: An open repository for developmental vocabulary data.
\emph{Journal of Child Language}, \emph{44}(3), 677.

\leavevmode\vadjust pre{\hypertarget{ref-frank2021}{}}%
Frank, M. C., Braginsky, M., Yurovsky, D., \& Marchman, V. A. (2021).
\emph{Variability and consistency in early language learning: The
wordbank project}. MIT Press.

\leavevmode\vadjust pre{\hypertarget{ref-Makransky2016}{}}%
Makransky, G., Dale, P. S., Havmose, P., \& Bleses, D. (2016). An item
response theory-based, computerized adaptive testing version of the
{M}ac{A}rthur-{B}ates {C}ommunicative {D}evelopment {I}nventory: {W}ords
\& {S}entences ({CDI:WS}). \emph{Journal of Speech, Language, and
Hearing Research}, \emph{59}(2), 281--289.

\leavevmode\vadjust pre{\hypertarget{ref-mayor2019}{}}%
Mayor, J., \& Mani, N. (2019). A short version of the
{M}ac{A}rthur-{B}ates {C}ommunicative {D}evelopment {I}nventories with
high validity. \emph{Behavior Research Methods}, \emph{51}(5),
2248--2255.

\end{CSLReferences}

\bibliographystyle{apacite}


\end{document}
